{\rtf1\ansi\ansicpg1252\cocoartf2513
\cocoatextscaling0\cocoaplatform0{\fonttbl\f0\fswiss\fcharset0 Helvetica;}
{\colortbl;\red255\green255\blue255;}
{\*\expandedcolortbl;;}
\paperw11900\paperh16840\margl1440\margr1440\vieww10800\viewh8400\viewkind0
\pard\tx566\tx1133\tx1700\tx2267\tx2834\tx3401\tx3968\tx4535\tx5102\tx5669\tx6236\tx6803\pardirnatural\partightenfactor0

\f0\fs24 \cf0 \\documentclass[11pt,a4paper]\{article\}\
\
\\usepackage[utf8]\{inputenc\}\
\\usepackage[greek,english]\{babel\}\
\\usepackage\{alphabeta\} \
\\usepackage\{float\}\
\
\\usepackage[pdftex]\{graphicx\}\
\\usepackage[top=0.8in, bottom=1in, left=0.8in, right=0.8in]\{geometry\}\
\
\\setlength\{\\parskip\}\{8pt plus2pt minus2pt\}\
\
\\widowpenalty 10000\
\\clubpenalty 10000\
\
\\newcommand\{\\eat\}[1]\{\}\
\\newcommand\{\\HRule\}\{\\rule\{\\linewidth\}\{0.5mm\}\}\
\
\\usepackage[official]\{eurosym\}\
\\usepackage\{enumitem\}\
\\setlist\{nolistsep,noitemsep\}\
\\usepackage[hidelinks]\{hyperref\}\
\\usepackage\{cite\}\
\\usepackage\{lipsum\}\
\\begin\{document\}\
\
%===========================================================\
\\begin\{titlepage\}\
\\begin\{center\}\
\
% Top \
\\includegraphics[width=0.30\\textwidth]\{UClogo.jpg\}~\\\\[1cm]\
\
% Title\
\\HRule \\\\[0.4cm]\
\{ \\LARGE \
  \\textbf\{ENEL471 Power Electronics 2 Assignment\}\\\\[0.4cm]\
  \\emph\{DC Motor Control for a  Go-Cart\}\\\\[0.4cm]\
\}\
\\HRule \\\\[1.5cm]\
\
% Author\
\{ \\large\
  Matt Blake (ID: 58979250) \\\\[0.5cm]\
  Josie Williamson (ID: 65029169) \\\\[0.5cm]\
  Brendain Hennessy (ID: 57190084) \\\\[2cm]\
\
  \\textbf\{Group 8\}\\\\ [1cm]\
  \}\
\
\\vfill\
\
%\\textsc\{\\Large Cyprus University of Technology\}\\\\[0.4cm]\
\\textsc\{\\large Department of Electrical and Computer Engineering\}\\\\[0.4cm]\
% Bottom\
\{\\large \\today\}\
 \
\\end\{center\}\
\\end\{titlepage\}\
\\newpage\
\
%===========================================================\
\\tableofcontents\
\\addtocontents\{toc\}\{\\protect\\thispagestyle\{empty\}\}\
\\newpage\
\\setcounter\{page\}\{1\}\
\
%===========================================================\
%===========================================================\
\\section\{Introduction\}\\label\{sec:intro\}\
\
Electrical motors used in vehicles is expanding exponentially worldwide, with the worldwide electrical vehicle count increasing from 20 thousand in 2010 to 4.79 million in 2019 \\cite\{EnergyAgency\}. These motors require a control system to facilitate stable and controllable operation. However, voltage-based control systems have safety issues during short circuit faults. Therefore, this report discusses the design of a fixed-frequency current-mode controller to drive a go-cart\'92s DC motor. This includes a discussion on current-mode control, as well as documenting the circuit\'92s operation and design choices.\
\
\\section\{Current-Mode Control Circuit Design\}\\label\{sec:Circuit-design\}\
Current mode control is a form of motor control that involves sensing the current of a system. This is usually embedded within a slower acting voltage-mode control loop. Acting immediately to changes in output current allows the current to be kept constant \\cite\{voltagewebsite\}.\
\
If only voltage-mode control is used, short-circuit faults will cause large current flows. This can cause component damage and potential harm to people. However, as current-mode control acts by sensing the output current, it can quickly respond to faults \\cite\{slopecomp\}. This increases the safety of the design.\
\
An issue with current-mode control is that there may be noise in the go-cart system, depending on external factors such as unstable terrain. This can create a current perturbation, \\Delta I as the current requirements will also depend on these external factors. Figure 1 shows how the average current may vary with the duty cycle, while the peak (regulated) current remains constant \\cite\{slopecomp\}. This results in a difference of \\Delta t and the greater this current perturbation, the greater the time difference. This difference between peak and average current is greatest when the duty cycle is greater than 50\\%, as the error will keep increasing for each charge/discharge cycle of the current. This will lead to controller sub-harmonic oscillations causing control instability \\cite\{journal1\}. To fix this instability a slope compensation circuit is used. To do this, a scaled version of the timing circuit\'92s sawtooth waveform is added to the measured current waveform \\cite\{voltagewebsite\}. This will decrease the gradient of the descending slope, fixing the instability problem. When correctly implemented, this slope compensation circuit will address this issue of instability at duty cycles above 50\\%, and the average current should remain constant despite changes in the duty cycle \\cite\{slopecomp\}.\
\
\\begin\{figure\}[H]\
    \\centering\
    \\includegraphics[scale = 0.8]\{currentmodecontrol.PNG\}\
    \\caption\{Current perturbation in a system.\}\
    \\text\{Source: \\cite\{slopecomp\}\}\
\\end\{figure\}\
\
\\section\{Design and Implementation\}\\label\{sec:design\}\
This section discusses how the fixed-frequency current-mode controller was designed to drive a DC motor using pulse width modulation (PWM). This design was based on the UC3843 current mode PWM controller \\cite\{UC3843Datasheet\}. Operational amplifiers (opamps) were used to process the circuit\'92s input signals. Regulation circuitry was used to ensure components had a clean power supply. A control system was designed to provide the go-cart\'92s driver with smooth control over the go-cart. The completed circuit is shown in Appendix A.\
\
\\subsection\{Design Constraints\}\
\
There were several design constraints that the go-cart design had to comply with. These constraints were implemented to protect the electronic components, as well as the driver of the go-cart. These design constraints are listed below:\
\\begin\{enumerate\}\
    \\item The forward motor current must not exceed 350 A.\
    \\item The current-time area above 150 A must not be exceeded for more than 100 As.\
    \\item The average forward motor current must not exceed 150 A.\
    \\item The reverse motor current must not exceed 100 A.\
    \\item No MOSFET can be switched faster than 5 kHz.\
    \\item PWM must be controllable via the throttle and be shut off when the brake is pressed.\
    \
    \\item PWM with a voltage swing of 5 V referenced to ground.\
    \\item PWM must have a rise and fall time smaller than 1 \\mu s.\
    \\item Utilise only a single supply voltage of 24 V.\
    \\item PWM must have a switching frequency of 2.5kHz\\pm10\\%\
    \
\\end\{enumerate\}\
If the current limits or the MOSFET switching speed constraint is exceeded, the go-cart will enter a 5-second shutdown. \
\
\\subsection\{PWM Signal Generation\}\
The UC3843 chip was used to produce a PWM (pulse width modulation) signal. The frequency of this PWM signal was set by creating a sawtooth waveform on the Rt/Ct pin of the UC3843. The resistor, \\(R_T\\), and capacitor, \\(C_T\\) attached to the pin, set the frequency, \\(f\\), of the waveform by \\cite\{UC3843Datasheet\} which can be seen in Eq. 1. \
\\begin\{equation\}\
f = \\frac\{1.72\}\{R_T C_T\}\
\\end\{equation\}\
The maximum switching frequency is 5 kHz and the desired PWM switching frequency is 2.5kHz, as given by design constraint 5 and 10. Therefore, a 2.5 kHz switching frequency was chosen.  A 100 nF capacitor was used as 100 nF is a standard E series component value \\cite\{standardvalues\} and is within the UC3843\'92s recommended range \\cite\{UC3843Datasheet\}. Based on Eq. 1, a resistance of 6880 \uc0\u937  is needed to generate a 2.5 kHz sawtooth. This was created using standard E12 component values \\cite\{standardvalues\} through the series combination of a 6.8 k\u937  and an 82 \u937  resistor. This allowed a sawtooth waveform to be produced, as shown in Figure 2, which in practice gave a frequency value of 2.56 kHz, meeting design constraint 10.\
\
\\begin\{figure\}[H]\
    \\centering\
    \\includegraphics[scale = 0.7]\{sawtooth.PNG\}\
    \\caption\{The sawtooth waveform produced at the UC3843's \\(R_T/C_T\\) pin\}\
\\end\{figure\}\
\
To meet the requirements of the IRS21844 motor driving being controlled by the output PWM signal, the PWM signal was inverted. An NMOS inverter was used to achieve this, as shown in Figure 3, as NMOS inverters only require a single sided supply, as per design constraint 8. \
\
\\begin\{figure\}[H]\
    \\centering\
    \\includegraphics[scale = 0.4]\{NMOSinverter.PNG\}\
    \\caption\{NMOS Inverter used for the PWM output\}\
\\end\{figure\}\
\
\
To fulfil constraint 5, the output was pulled up to 5 V. A 100 k\uc0\u937  pull up resistor was used to current and therefore, the voltage drop across the output. The time constant, \u964 , of a resistor-capacitor (RC) circuit is given by \\cite\{RCcharging\}, shown below in Eq. 2: \
\
\\begin\{equation\}\
    \\tau = RC\
\\end\{equation\}\
Where R is the resistance of the circuit and C is the capacitance. The BSH105,215 transistor used has 152 pF input capacitance \\cite\{LEMHAIS\}. Therefore, a 1 k\uc0\u937  bleed resistor was used to provide a 152 ns RC time constant, allowing the PWM to be switched a 1 \\mu s when switching the output, as per design requirement 8.  \
\
The NMOS signal successfully inverted the PWM signal produced by the UC3843\'92s output pin, as shown in Figure 4. This signal had a swing of 5 V, as per design requirement 7. It also had a maximum rise and fall times time of 430 ns, faster than the 1 \'b5s given by design requirement 8.\
\
\\begin\{figure\}[H]\
    \\centering\
    \\includegraphics[scale = 0.8]\{PWMoutputsignal.PNG\}\
    \\caption\{The produced PWM output signal\}\
\\end\{figure\}\
\
\\subsection\{Motor Voltage-Control Loop\}\
An outer voltage-control loop is used for primary control of the motor. This voltage control loop is slow acting and cannot respond instantly to faults. It acts on the user-controlled accelerator input and a voltage representing the motor current.\
The motor current, \\(I_o\\), is measured by a LEM HIAS 150P current sensor. This produces a voltage, \\(V_o\\), which is given by \\cite\{LEMHAIS\}, which can be seen in Eq. 3:\
\\begin\{equation\}\
 V_o= 2.5 \\pm 0.625 ( \\frac\{I_o\}\{I_\{nom\}\})\
\\end\{equation\}\
Where \\(I_\{nom\}\\), is the nominal root mean squared output current, which will be 150 A as per design constraint 3. This voltage signal required filtering to eliminate any ripple from the motor drivers. The PWM switching frequency was 2.5 kHz, thus, this low pass filter was created to have a cut-off frequency of 10x lower than the switching frequency at 250 Hz to properly attenuate ripple. Therefore, an active inverting low-pass filter (LPF) was implemented using LMC6484 single supply opamps, as shown in Figure 5. \
\
\\begin\{figure\}[H]\
    \\centering\
    \\includegraphics[scale = 1]\{250HzLPF.PNG\}\
    \\caption\{Output current low pass filter\}\
    \\label\{fig:my_label\}\
\\end\{figure\}\
\
This implementation was chosen to buffer the signal and as it only requires one supply, as per design constraint 9. The cut-off frequency, \\(f_c\\), of a non-inverting LPF filter is given by \\cite\{LPF\}, which can be seen in Eq. 4:\
\\begin\{equation\}\
f_c = \\frac\{1\}\{2 \\pi R C\}\
\\end\{equation\}\
Where R is an input resistor and C is an input shunt capacitor. A standard valued 47 nF capacitor was used \\cite\{standardvalues\}. Therefore, to achieve a 250 Hz cut-off frequency, a resistance of 13.5 k\uc0\u937  is needed. This was achieved with standard valued components by placing a 12 k\u937  resistor in series with a 1.5 k\u937  resistor. This LPF was designed with a gain of 1, to avoid amplifying or attenuating the signal. This was done by placing the opamp\'92s feedback loop in a voltage follower configuration. This successfully smoothed the LEM sensor\'92s ripple, as shown in Figure 6.\
\
\\begin\{figure\}[H]\
    \\centering\
    \\includegraphics[scale = 0.8]\{LEMvsfiltered.PNG\}\
    \\caption\{The LEM sensor output (orange) plotted against the filtered LEM signal (green)\}\
    \\label\{fig:my_label\}\
\\end\{figure\}\
\
The acceleration is controlled by a linear potentiometer which operates roughly from 5-10 k\uc0\u937 . The potentiometer is supplied between 5 V and ground to allow for a large range of control. As the potentiometer value is unknown the acceleration output needs to be buffered using a unity gain operational amplifier. This is so that the unknown Th\'e9venin resistance of the potentiometer does not affect downstream devices.\
\
To control the motor, an error signal between the desired acceleration and the actual acceleration was used. The measured voltage, given in Eq. 3, is subtracted from the acceleration input using a voltage summation opamp configuration, as shown in Figure 7. A 2.5 V has to be added to this sum because the internal UC3843 error comparator is compared to 2.5 V\\cite\{UC3843Datasheet\}. \
\
\\begin\{figure\}[H]\
    \\centering\
    \\includegraphics[scale = 1]\{summationamp.PNG\}\
    \\caption\{The voltage loop summation amplifier used for error calculation\}\
    \\label\{fig:my_label\}\
\\end\{figure\}\
\
\
Proportional and integral (PI) control were used to calculate the output PWM duty cycle. These methods were used to provide a fast response and eliminate steady-state error \\cite\{PIcontrol\}. This was implemented by connecting the UC3843\'92s internal error amplifier into an integrator configuration. The time constant of this integrator was set to equal the time constant of the LPF used to smooth the output voltage. This provides pole-zero cancellation of the control transfer function, increasing the stability of the system \\cite\{PoleZero\}. The LPF has a time constant of 635 \\mu s. Therefore, the integral gain, \\(K_i\\), is given by:\
\\begin\{equation\}\
K_i = \\frac\{1\}\{R_i C\} = \\frac\{1\}\{\\tau\}\
\\end\{equation\}\
\
Where \\(R_i\\), is the integrator\'92s input resistor and C is the feedback capacitor. This gives an integral gain of 15.74, however, this was reduced to a gain of 2, as we had found this this gain value was too large. The proportional gain, \\(K_p\\), is given by the DC gain of the integrator, which is calculated using:\
\
\\begin\{equation\}\
    K_p = K_i\\tau =  \\frac\{R_f\}\{R_i\} \
\\end\{equation\}\
\
Where \\(R_f\\), is the integrator\'92s feedback resistor. A proportional gain of 4 was iteratively chosen to provide a stable response. However, this should increase to provide a faster response to changes, as current the system is over-reliant on integral control. These gains were created using standard E12 component values of 10 \'b5F, 200 k\uc0\u937  and 51 k\u937  for the capacitor, feedback resistor and input resistor respectively \\cite\{standardvalues\}.\
\
\\subsection\{Motor Current Control Loop\}\
An inner fast-acting current-control loop was implemented, to provide short circuit protection and fine control of the motor current. This involved feeding the sensed value of the motor current into the UC3843\'92s Isense pin. The sawtooth waveform from the UC3843\'92s Rt/Ct pin was used to prevent sub-harmonic instability by providing slope compensation, as discussed in Section 2.\
\
The UC3843 will shutdown if the voltage on \\(I_\{sense\}\\)  exceeds 1 V [5]. To avoid this, the signals were scaled using differential opamps. Differential opamps were used as they can produce a gain lower than 1 while using a single-sided supply, as required from design constraint 9. The gain of a differential amplifier, \\(A_v\\), is given by: \
\\begin\{equation\}\
    A_v= \\frac\{R_2\}\{R_1\}   \
\\end\{equation\}\
Where \\(R_1\\) represents the input resistor, as well as the shunt resistor connected to the feedback loop. \\(R_2\\) represents both the feedback resistor shunt resistor connected to the feedback. The 1 V maximum on the Isense pin was split into 0.8 V from the current sensor and of 0.2 V from the slope compensation circuit, to provide accurate current sensing while still having adequate slope compensation.\
\
With a maximum motor current of 350 A, as per design constraint 1, a maximum LEM sense voltage of 3.96 V will be produced, as defined by Eq. 3. Therefore, a gain of 0.2 is used to ensure that the LEM sensor only contributes 0.8 V to the summing amplifier output. Using a standard resistor value of 10 k\uc0\u937  for the feedback resistance, Eq. 7 shows that a resistance of 50 k\u937  is needed for the input. This was created with standard component values by placing two 100 k\u937  resistors in parallel \\cite\{standardvalues\}. This circuit is shown in Figure 8.\
\
\\begin\{figure\}[H]\
    \\centering\
    \\includegraphics[scale = 1.2]\{lemamplifier.PNG\}\
    \\caption\{LEM scaling amplifier\}\
    \\label\{fig:my_label\}\
\\end\{figure\}\
\
The timing waveform produced by the UC3843\'92s Rt/Ct pin has a maximum voltage of approximately 2.1V \\cite\{UC3843Datasheet\}. Therefore, a gain of 0.095 was needed to scale this to a maximum of 0.2 V. Using Eq. 7, shows that with a standard valued feedback resistor of 10 k\uc0\u937 , a 105.3 k\u937  input resistance will provide a gain of 0.095. This was created using a series combination of a 100 k\u937  and a 5.6 k\u937  resistor, which are standard E12 component values \\cite\{standardvalues\}.\
\
A summation amplifier was used to add the scaled timing and current waveforms. A non-inverting configuration was used, due to design constraint 9. Summation amplifiers have a gain, \\(A_v\\), for each input:\
\
\\begin\{equation\}\
    A_v =  \\frac\{R_F\}\{R_1\}   \
\\end\{equation\}\
\
Where \\(R_1\\) is the input resistor for that signal and \\(R_F\\) is the feedback resistor. This scaling could have been applied to mitigate the need for differential scaling amplifiers. However, the differential amplifiers also provided input buffering. As both the measured current and timing waveform are scaled by these differential amplifiers, the summation amplifier was chosen to have a gain of 1. Therefore, equal and E12 standard valued 10 k\uc0\u937  resistors were used for both inputs and the feedback loop \\cite\{standardvalues\}. This allowed the total compensated current waveform to have a maximum value of 1 V. The circuit is shown in Figure 9 and the compensated current waveform is shown in Figure 10.\
\\begin\{figure\}[H]\
    \\centering\
    \\includegraphics[scale = 1]\{isensesum.PNG\}\
    \\caption\{Current loop summation amplifier\}\
    \\label\{fig:my_label\}\
\\end\{figure\}\
\
\\begin\{figure\}[H]\
    \\centering\
    \\includegraphics[scale = 0.8]\{slopecompensated.PNG\}\
    \\caption\{The slope compensated current waveform fed into the UC3843\'92s  \\(I_\{sense\}\\) pin\}\
    \\label\{fig:my_label\}\
\\end\{figure\}\
\
The LEM sensor\'92s output voltage can spike due to high-frequency harmonics. This can cause false PWM triggering, as it is connected to \\(I_\{sense\}\\). Therefore, a 1 MHz Low-pass filter was implemented to remove these high-frequency components, shown in Figure 11. For simplicity, a passive resistor-capacitor filter was used, which has a cut-off frequency given by Eq. 4. Therefore, a 1 MHz cut-off frequency can be achieved through a 3.3 nF capacitor and a and 47 \uc0\u937  resistor, which are standard E12 component values \\cite\{standardvalues\}. These values give the filter a 155 ns time constant, which is appropriately far lower than the 250 Hz switching period of 400 \\mu s.\
\
\\begin\{figure\}[H]\
    \\centering\
    \\includegraphics[scale = 1.2]\{LPF.PNG\}\
    \\caption\{Low-pass filter used to condition the UC3843's \\(I_\{sense\}\\) pin\}\
    \\label\{fig:my_label\}\
\\end\{figure\}\
\
\\subsection\{Voltage Regulation\}\
The lithium-ion batteries supplied for this assignment produce a voltage of 24 V. This was stepped down to 15 V and to 5 V to power other circuit components. This was achieved using the MC7815 and MC7805 to produce 15 V and 5 V respectively, as shown in Figure 12. \
\
\\begin\{figure\}[H]\
    \\centering\
    \\includegraphics[scale = 0.8]\{voltageregulation.PNG\}\
    \\caption\{Voltage Regulator\}\
    \\label\{fig:my_label\}\
\\end\{figure\}\
\
Voltage regulators are useful for removing noise in the power supply. However, they have a limited frequency response and are better at lower frequencies \\cite\{VoltageReg\}. Thus, low-pass filtering of the power supply was implemented. RC filters were used to remove high frequencies noise from the power supply. These were created using the combination of ferrite beads and 100 nF decoupling capacitors. The regulator\'92s input voltage was also decoupled using a 330 nF capacitor. These capacitor values specified by the datasheet for decoupling and to improve the transient response of the regulator \\cite\{VoltageRegDatasheet\}.\
\
\\subsection\{Shutdown\}\
The brake switch was connected between ground and the IRS21844 motor driver\'92s shutdown pin, \\(( \\overline\{SD\} ) \\). This allowed the go-cart\'92s s driver to stop driving the motor and decelerate by pressing the brake switch and therefore pulling the \\(( \\overline\{SD\} ) \\)  pin low. \
An RC debouncer circuit, as shown in Figure 13, was implemented to provide a smooth braking response, by removing the bouncing from the brake switch\'92s mechanical contacts. \
\
\\begin\{figure\}[H]\
    \\centering\
    \\includegraphics[scale = 1]\{RCdebouncer.PNG\}\
    \\caption\{The RC debouncer used in the break circuit\}\
    \\label\{fig:my_label\}\
\\end\{figure\}\
\
This circuit works by using a capacitor to store the current state of the system. This capacitor has a non-zero time constant, meaning it will not be instantly charged or discharged due to the switch\'92s bouncing. The capacitor is charged to 5 V through \\(R_4\\) and \\(D_1\\), as shown in Figure 13, whenever the brake switch has not been pulled closed. When the brake switch is pushed, the capacitor discharges over \\(R_7\\) \\cite\{debouncing\}. The time constant of 39 ms was chosen as 20 ms is commonly needed to debounce mechanical switches \\cite\{debouncing\}. However, the exact debounce time of the switch used in the break was not known, so a 19 ms buffer was used. Therefore, using Eq. 2, standard E12 component values \\cite\{standardvalues\} of 100 nF and 39 k\uc0\u937  were chosen for the capacitor and the resistor respectively. 39 k\u937  was chosen for both the resistor in the discharging and discharging networks, to provide equal time constants. The diode used is a 200 mA BAS16TT1G diode \\cite\{diode\}. This current limit was adequate, as the maximum current with 39 k\u937  resistors is 128 \\mu A.\
\
An inverting Schmitt trigger was used to provide extra debouncing noise protection through its hysteretic comparator thresholds. These thresholds were biased around the midpoint voltage of 2.5 V and calculated based on the bias resistors. The upper threshold, \\(V_\{UT\}\\), was calculated using \\cite\{schmitttrigger\}, shown in Eq. 9: \
\
\\begin\{equation\}\
    V_\{UT\} = V_\{cc\}  \\frac\{R_I\}\{R_I+R_F\}\
\\end\{equation\}\
\
Where \\(V_\{cc\}\\) is the supply voltage of 15 V. \\(R_F\\) and \\(R_I\\) are the feedback resistors represented by \\(R_F\\) and \\(R_I\\) respectively. An \\(R_F/R_I\\)  ratio of 19 was chosen to give an upper threshold of 3.3 V and a lower threshold of 1.71 V, which provides adequate noise protection. Large resistor values were chosen so the feedback loop does not act as a voltage divider, reducing the capacitor voltage from 5 V. Therefore, standard component values of 18 M\uc0\u937  and 1 M\u937  \\cite\{standardvalues\} were used for \\(R_F\\) and \\(R_I\\) respectively.\
\
\\subsection\{Extra Design Considerations\}\
A printed circuit board (PCB) was laid out for the circuit, as shown in Figure 14. This reduced the circuits parasitic inductance by introducing traces, ground planes and a two-layer PCB was used, where the majority of components were placed on the top signal layer. The bottom layer was used for signal traces and small components. These two layers had ground planes, to minimise stray inductance, which were linked using stitching vias to allow for the shortest current return path.\
\
\\begin\{figure\}[H]\
    \\centering\
    \\includegraphics[scale = 0.8]\{PCB.PNG\}\
    \\caption\{The PCB used for the current mode controller board\}\
    \\label\{fig:my_label\}\
\\end\{figure\}\
\
Decoupling capacitors were used throughout the board to reduce the loop area of parasitic inductance. These were placed on critical signals and power supplies. The PCB\'92s bottom layer was used to place decoupling capacitors as close to the pins they are decoupling as possible.\
\
Test points were added to the PCB to allow the measurement of signals when testing a debugging the board. Multiple ground points were added to detect differences in voltage at different points on the ground plane. Similarly, 0 \uc0\u937  resistors were placed in series with components. This allowed the components to be isolated and individually tested.\
\
A Wurth Electronics 150060RS75000 red light-emitting diode (LED) was added to indicate that the UC3843 is powered on \\cite\{LEDDatasheet\}. This was implemented to aid debugging. The current through a diode is given by \\cite\{LED\}, shown by Eq. 10: \
\
\\begin\{equation\}\
    I = \\frac\{V_\{cc\}-V_f\}\{R\}\
\\end\{equation\}\
\
Where \\(V_cc\\) is the 15 V power supply from the voltage regulator, \\(V_f\\) is the LED\'92s 2 V forward voltage drop and R is a series resistance. The current to 1.9 mA, which provides a reasonable brightness. Therefore, it was connected in series with a 6.8 k\uc0\u937  pull-down resistor. \
\
\\subsection\{Future Improvements\}\
Currently, the designed PCB protects against power supply reverse polarity by allowing the power supply header to be only inserted in one direction. However, short circuits can occur if the header is soldered on backwards or if a fault occurs in the circuit. This can lead to component damage. To prevent this, a reverse polarity protection circuit could be connected to the power supply header. This could be done with a P-channel MOSFET that turns off when a Zener diode becomes reverse biased due to reverse polarity \\cite\{ReversePolarity\}. A fuse could additionally be used for added short circuit protection.\
\
The stray inductance of the circuit could be reduced to reduce noise in the signals and power connections. A four-layer PCB could be used to do this by adding ground and power planes. These planes allow trace widths to be reduced \\cite\{ParasiticCap\} and provide interplane capacitance for high-frequency filtering. However, these additional layers may increase the manufacturing cost of the board \\cite\{PCBcost\}.\
\
\
\\section\{Conclusion\}\\label\{sec:res\}\
This project involved designing and building a fixed-frequency current-mode controller to drive an electric go-cart. This design was based on the using the UC3843 current control chip. Slope compensation was implemented, allowing the control to work stability duty cycles above 50\\%. \
\
The completed circuit successfully meet the design requirements. A 5 V, 2.56 kHz PWM waveform, with maximum rise and fall times of 430 ns, was successfully produced. This PWM output was controllable through the go-cart's throttle and break. However, a slow response was observed due to an over-reliance on integral gain, which could be rectified by increasing the proportional gain. Furthermore, signal quality in the board could be improved by minimising stray inductances, such as by reducing the length of signal and power traces.\
\
\\newpage\
\
\
%===========================================================\
%===========================================================\
\
\
\\bibliographystyle\{IEEEtran\}\
\\bibliography\{refs\}\
\
\
\\newpage\
\
\\section\{Appendices\}\
\\subsection\{Appendix A - Circuit Schematic\}\
A circuit diagram of the designed current-mode control circuit was created during development and used to track changes and aid in debugging. The completed schematic is shown in Figure 15.\
\
\\begin\{figure\}[H]\
    \\centering\
    \\includegraphics[scale = 1]\{schematic.PNG\}\
    \\caption\{Circuit Schematic of the Current Mode Control Circuit\}\
    \\label\{fig:my_label\}\
\\end\{figure\}\
\
\\subsection\{Appendix B - PCB Layout\}\
The printed circuit board was laid out using Altium Designer. A two-layer PCB was used, where the majority of components were placed on the top signal layer as shown in Figure 16. The bottom layer was used for signal traces and placing decoupling capacitors as close to the pins they are decoupling as possible, as shown in Figure 17. These two layers had ground planes, to minimise stray inductances, which were linked using stitching vias.\
\
\\begin\{figure\}[H]\
    \\centering\
    \\includegraphics[scale = 1]\{PCBtopside.PNG\}\
    \\caption\{Top side of the current mode controller PCB\}\
    \\label\{fig:my_label\}\
\\end\{figure\}\
\
\\begin\{figure\}[H]\
    \\centering\
    \\includegraphics[scale = 1]\{PCBbottomside.PNG\}\
    \\caption\{Bottom side of the current mode controller PCB\}\
    \\label\{fig:my_label\}\
\\end\{figure\}\
\
\\subsection\{Appendix C - Bill of Materials\}\
The components used in their design, as well as their footprint are shown in Table 1.\
\
\
\\begin\{figure\}[H]\
    \\centering\
    \\text\{Table 1: Bill of Materials\} \\\\\
    \\includegraphics[scale = 1]\{billofmaterials.PNG\}\
\\end\{figure\}\
\
\\begin\{figure\}[H]\
    \\centering\
    \\includegraphics[scale = 1]\{billofmaterials2.PNG\}\
\\end\{figure\}\
\
\
\\end\{document\} \
}